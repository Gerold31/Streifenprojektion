%\documentclass[ngerman,a4paper,parskip=half]{scrartcl}
\documentclass[ngerman,a4paper]{scrartcl}

\usepackage[ngerman]{babel}
\usepackage[utf8]{inputenc}
%\usepackage[left=3cm,right=4cm,top=3cm,bottom=6cm,includeheadfoot]{geometry}
\usepackage[onehalfspacing]{setspace}
\setlength{\parskip}{1em}

\usepackage{helvet}
\usepackage[T1]{fontenc}
\usepackage{amsmath, amssymb, amstext}
\usepackage[affil-it]{authblk}
\usepackage[round]{natbib}
\usepackage[nolist,footnote]{acronym}
\usepackage{graphicx}
\usepackage{xcolor}

\def \N{\mathbb{N}}
\def \Z{\mathbb{Z}}
\def \Q{\mathbb{Q}}
\def \R{\mathbb{R}}
\def \C{\mathbb{C}}
\def \fov{\mathrm{fov}}

\begin{acronym}[FOV]
	\acro{FOV}{Field of view}
\end{acronym}

\title%[Streifenlichtprojektion]
{
	Streifenlichtprojektion und optische Analyse zur Oberflächeninspektion
}

\author[D. Wagner, J. Spangenberg, L. Kramer]
{
	Dennis~Wagner,
	Johannes~Spangenberg,
	Leroy~Kramer
}

\date{\today}
%\subject{Informatik}

%Kopf- und Fußzeile
\usepackage{fancyhdr}
\pagestyle{fancy}
\fancyhf{}

%Kopfzeile links bzw. innen
\fancyhead[L]{\nouppercase{\leftmark}}
%Kopfzeile rechts bzw. außen
\fancyhead[R]{\today}
%Linie oben
\renewcommand{\headrulewidth}{0.5pt}

%Fußzeile mittig
\fancyfoot[C]{\thepage}
%Linie unten
\renewcommand{\footrulewidth}{0.5pt}

\begin{document}

% ---------------------------------------------------------------------------- %

\definecolor{HUblue}{RGB}{0, 55, 108}

\begin{titlepage}
\begin{center}

%\colorbox{HUblue!30}{
\begin{minipage}{\textwidth}
	\begin{minipage}[c]{.8\textwidth}
		\textsc{\LARGE Humboldt-Universität zu Berlin}
		
		Institut für Informatik\\
		Lehrstuhl Signalverarbeitung und Mustererkennung
	\end{minipage}\hfill
	\begin{minipage}[c]{.2\textwidth}
		\includegraphics{husiegel}
	\end{minipage}
\end{minipage}
%}
\vspace{1.5cm}

\textsc{\Large Semesterprojekt}\\[0.5cm]

% Title
\newcommand{\HRule}{\rule{\linewidth}{0.5mm}}
\HRule \\[0.4cm]
{\huge \bfseries Streifenlichtprojektion und optische Analyse zur Oberflächeninspektion}
\HRule \\[1.5cm]

% Author and supervisor
\begin{minipage}{0.4\textwidth}
\begin{flushleft} \large
\emph{Autoren:}\\
Dennis~Wagner,\\
Johannes~Spangenberg,\\
Leroy~Kramer
\end{flushleft}
\end{minipage}
\hfill
\begin{minipage}{0.4\textwidth}
\begin{flushright} \large
\emph{Betreuende Hochschullehrerin:} \\
Prof.~Dr.~Meffert
\end{flushright}
\end{minipage}

\vfill

% Unterer Teil der Seite
{\large \today}

\end{center}
\end{titlepage}

\tableofcontents
\newpage

% ---------------------------------------------------------------------------- %

\section{Einleitung}

In verschiedenen Fällen ist es hilfreich oder notwendig ein \emph{komplexes} dreidimensionales Objekt zu vermessen. Aus solchen Vermessungen resultierende Modelle können in der Unterhaltungsindustrie für die Film- und Spielproduktion verwendet werden. Außerdem ermöglichen Verfahren zur Vermessung von Geometrien automatisierte Qualitätskontrollen und neue Methoden zur automatisierten Fertigung oder Verarbeitung.

In unserem Projekt arbeiten wir mit der \emph{Streifenlichtprojektion}. Dabei projektieren wir einen Streifen auf eine Oberfläche und versuchen aus einer Aufzeichnung der projektierten Linie die Form der Struktur zu rekonstruieren.

% ---------------------------------------------------------------------------- %

\section{Theoretische und technische Grundlagen}

Es haben sich in den letzten Jahrzehnten viele verschiedene Methoden entwickelt, mit denen man dreidimensionale Strukturen der realen Welt vermessen kann. So existieren zusätzlich zur Streifenlichtprojektion beispielsweise Verfahren mit \emph{Stereo-Vision}, \emph{Structure from Motion}, \emph{Shape from Shading} und \emph{Time of Flight}.

\subsection{Perspektivische Projektion}

% ---------------------------------------------------------------------------- %

\section{Aufbau und Algorithmen}

\subsection{Hardware}

Die Hardware besteht aus einer Webcam und einem Linienlaser, der auf einem Modellbauservo montiert ist. Laser und Servo werden von einem ''Spark Core'' Mikrocontroller gesteuert, welcher wiederum über USB von der Software gesteuert wird.

\subsection{Grundlegende Softwarearchitektur}

\subsection{Klassen und Funktionsüberblick}

\subsection{Ansteuerung der Hardware}

Die Software steuert die Hardware über USB, wobei sich der Mikrocontroller als virtuellen COM-Port ausgibt. Zur Kommunikation wird ein einfaches Protokoll verwendet, das 2 Byte lange Nachrichten an den Mikrocontroller sendet. Das erste Byte gibt den Befehl an, das zweite den Parameter.\\

\begin{tabular}{|c|c|c|}
\hline
1. Byte & 2. Byte & Beschreibung \\
\hline
'm'\footnotemark & $\alpha \in [0,180]$ & Setzt die Servoposition auf $\alpha ^\circ$.\\
\hline
'l' & '0' oder '1' & Schaltet den Laser an ('1') bzw. aus ('0')\\
\hline
\end{tabular}

\footnotetext{Zeichen in Anführungszeichen stehen für den ASCII-Wert des Zeichens}


\subsection{Linienerkennung}

\subsubsection{Differenzbildung (Diff)}

\subsubsection{Farbfilter (Free)}

\subsection{Rekonstruktion}

% ---------------------------------------------------------------------------- %

\section{Auswertung}

% ---------------------------------------------------------------------------- %

\section{Zusammenfassung}

% ---------------------------------------------------------------------------- %

\section{Quellenverzeichnis}

% ---------------------------------------------------------------------------- %

\end{document}
